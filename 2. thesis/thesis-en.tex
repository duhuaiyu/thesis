\documentclass[a4paper,11pt,twoside]{report}
% THIS FILE SHOULD BE COMPILED BY pdfLaTeX

% ----------------------   PREAMBLE PART ------------------------------

% ------------------------ ENCODING & LANGUAGES ----------------------

\usepackage[utf8]{inputenc}
%\usepackage[MeX]{polski} % Not needed unless You have a name with polish symbols or sth
\usepackage[T1]{fontenc}
\usepackage[english,polish]{babel}


\usepackage{amsmath, amsfonts, amsthm, latexsym} % MOSTLY MATHEMATICAL SYMBOLS

\usepackage[final]{pdfpages} % INPUTING TITLE PDF PAGE - GENERATE IT FIRST!
%\usepackage[backend=bibtex, style=verbose-trad2]{biblatex}


\usepackage{commath} % various commands which can make writing math expressions easier --- documentation available at: https://ctan.gust.org.pl/tex-archive/macros/latex/contrib/commath/commath.pdf

\usepackage[hidelinks]{hyperref} % for hyperlinks, for example, urls, references to equations, entries in a bibliography --- hidelinks option removes rectangles around hiperlinks


% ---------------- MARGINS, INDENTATION, LINESPREAD ------------------

\usepackage[inner=20mm, outer=20mm, bindingoffset=10mm, top=25mm, bottom=25mm]{geometry} % MARGINS


\linespread{1.5}
\allowdisplaybreaks         % ALLOWS BREAKING PAGE IN MATH MODE

\usepackage{indentfirst}    % IT MAKES THE FIRST PARAGRAPH INDENTED; NOT NEEDED
\setlength{\parindent}{5mm} % WIDTH OF AN INDENTATION


%---------------- RUNNING HEAD - CHAPTER NAMES, PAGE NUMBERS ETC. -------------------

\usepackage{fancyhdr}
\pagestyle{fancy}
\fancyhf{}
% PAGINATION: LEFT ALIGNMENT ON EVEN PAGES, RIGHT ALIGNMENT ON ODD PAGES 
\fancyfoot[LE,RO]{\thepage} 
% RIGHT HEADER: zawartość \rightmark do lewego, wewnętrznego (marginesu) 
\fancyhead[LO]{\sc \nouppercase{\rightmark}}
% lewa pagina: zawartość \leftmark do prawego, wewnętrznego (marginesu) 
\fancyhead[RE]{\sc \leftmark}

\renewcommand{\chaptermark}[1]{\markboth{\thechapter.\ #1}{}}

% HEAD RULE - IT'S A LINE WHICH SEPARATES HEADER AND FOOTER FROM CONTENT
\renewcommand{\headrulewidth}{0 pt} % 0 MEANS NO RULE, 0.5 MEANS FINE RULE, THE BIGGER VALUE THE THICKER RULE


\fancypagestyle{plain}{
  \fancyhf{}
  \fancyfoot[LE,RO]{\thepage}
  
  \renewcommand{\headrulewidth}{0pt}
  \renewcommand{\footrulewidth}{0.0pt}
}



% --------------------------- CHAPTER HEADERS ---------------------

\usepackage{titlesec}
\titleformat{\chapter}
  {\normalfont\Large \bfseries}
  {\thechapter.}{1ex}{\Large}

\titleformat{\section}
  {\normalfont\large\bfseries}
  {\thesection.}{1ex}{}
\titlespacing{\section}{0pt}{30pt}{20pt} 

    
\titleformat{\subsection}
  {\normalfont \bfseries}
  {\thesubsection.}{1ex}{}


% ----------------------- TABLE OF CONTENTS SETUP ---------------------------

\def\cleardoublepage{\clearpage\if@twoside
\ifodd\c@page\else\hbox{}\thispagestyle{empty}\newpage
\if@twocolumn\hbox{}\newpage\fi\fi\fi}


% THIS MAKES DOTS IN TOC FOR CHAPTERS
\usepackage{etoolbox}
\makeatletter
\patchcmd{\l@chapter}
  {\hfil}
  {\leaders\hbox{\normalfont$\m@th\mkern \@dotsep mu\hbox{.}\mkern \@dotsep mu$}\hfill}
  {}{}
\makeatother

\usepackage{titletoc}
\makeatletter
\titlecontents{chapter}% <section-type>
  [0pt]% <left>
  {}% <above-code>
  {\bfseries \thecontentslabel.\quad}% <numbered-entry-format>
  {\bfseries}% <numberless-entry-format>
  {\bfseries\leaders\hbox{\normalfont$\m@th\mkern \@dotsep mu\hbox{.}\mkern \@dotsep mu$}\hfill\contentspage}% <filler-page-format>

\titlecontents{section}
  [1em]
  {}
  {\thecontentslabel.\quad}
  {}
  {\leaders\hbox{\normalfont$\m@th\mkern \@dotsep mu\hbox{.}\mkern \@dotsep mu$}\hfill\contentspage}

\titlecontents{subsection}
  [2em]
  {}
  {\thecontentslabel.\quad}
  {}
  {\leaders\hbox{\normalfont$\m@th\mkern \@dotsep mu\hbox{.}\mkern \@dotsep mu$}\hfill\contentspage}
\makeatother



% ---------------------- TABLES AD FIGURES NUMBERING ----------------------

\renewcommand*{\thetable}{\arabic{chapter}.\arabic{table}}
\renewcommand*{\thefigure}{\arabic{chapter}.\arabic{figure}}


% ------------- DEFINING ENVIRONMENTS FOR THEOREMS, DEFINITIONS ETC. ---------------

\makeatletter
\newtheoremstyle{definition}
{3ex}%                           % Space above
{3ex}%                           % Space below
{\upshape}%                      % Body font
{}%                              % Indent amount
{\bfseries}%                     % Theorem head font
{.}%                             % Punctuation after theorem head
{.5em}%                          % Space after theorem head, ' ', or \newline
{\thmname{#1}\thmnumber{ #2}\thmnote{ (#3)}}
\makeatother

\theoremstyle{definition}
\newtheorem{theorem}{Theorem}[chapter]
\newtheorem{lemma}[theorem]{Lemma}
\newtheorem{example}[theorem]{Example}
\newtheorem{proposition}[theorem]{Proposition}
\newtheorem{corollary}[theorem]{Corollary}
\newtheorem{definition}[theorem]{Definition}
\newtheorem{remark}[theorem]{Remark}

% --------------------- END OF PREAMBLE PART (MOSTLY) --------------------------





% -------------------------- USER SETTINGS ---------------------------

\newcommand{\tytul}{POLISH TITLE}
\renewcommand{\title}{ENGLISH TITLE}
\newcommand{\type}{Master} % Master OR Engineer
\newcommand{\supervisor}{dr inż. Promotor X} % TITLE AND NAME OF THE SUPERVISOR



\begin{document}
\sloppy
\selectlanguage{english}

\includepdf[pages=-]{titlepage-msc-en} % THIS INPUTS THE TITLE PAGE

\null\thispagestyle{empty}\newpage

% ------------------ PAGE WITH SIGNATURES --------------------------------

%\thispagestyle{empty}\newpage
%\null
%
%\vfill
%
%\begin{center}
%\begin{tabular}[t]{ccc}
%............................................. & \hspace*{100pt} & .............................................\\
%supervisor's signature & \hspace*{100pt} & author's signature
%\end{tabular}
%\end{center}
%


% ---------------------------- ABSTRACTS -----------------------------

{  \fontsize{12}{14} \selectfont
\begin{abstract}

\begin{center}
\title
\end{center}
Reinforcement learning (RL) is one of three basic machine learning paradigms, alongside supervised learning and unsupervised learning. Reinforcement learning algorithms have become very popular in the field of simple computer games, and games like chess, GO, and Atari have become testbeds of testing deep reinforcement learning algorithms. However, classical arcade fighting game would be a challenging because of the complexity of the command system and combo system. Expect basic moves such as jump, kick  \\

\noindent \textbf{Keywords:} Reinforcement learning, KOF97,Arcade game
\end{abstract}
}

\null\thispagestyle{empty}\newpage


{\selectlanguage{polish} \fontsize{12}{14}\selectfont
\begin{abstract}

\begin{center}
\tytul
\end{center}

TODO.\\

\noindent \textbf{Słowa kluczowe:} slowo1, slowo2, ...
\end{abstract}
}


%% --------------------------- DECLARATIONS ------------------------------------
%
%%
%%	IT IS NECESSARY OT ATTACH FILLED-OUT AUTORSHIP DEECLRATION. SCAN (IN PDF FORMAT) NEEDS TO BE PLACED IN scans FOLDER AND IT SHOULD BE CALLED, FOR EXAMPLE, DECLARATION_OF_AUTORSHIP.PDF. IF THE FILENAME OR FILEPATH IS DIFFERENT, THE FILEPATH IN THE NEXT COMMAND HAS TO BE ADJUSTED ACCORDINGLY.
%%
%%	command attacging the declarations of autorship
%%
%\includepdf[pages=-]{scans/declaration-of-autorship}
%\null\thispagestyle{empty}\newpage
%
%% optional declaration
%%
%%	command attaching the declaataration on granting a license
%%
%\includepdf[pages=-]{scans/declaration-on-granting-a-license}
%%
%%	.tex corresponding to the above PDF files are present in the 3. declarations folder 
%
\null\thispagestyle{empty}\newpage
% ------------------- TABLE OF CONTENTS ---------------------
% \selectlanguage{english} - for English
\pagenumbering{gobble}
\tableofcontents
\thispagestyle{empty}
\newpage % IF YOU HAVE EVEN QUANTITY OD PAGES OF TOC, THEN REMOVE IT OR ADD \null\newpage FOR DOUBLE BLANK PAGE BEFORE INTRODUCTION


% -------------------- THE BODY OF THE THESIS --------------------------------

\null\thispagestyle{empty}\newpage
\pagestyle{fancy}
\pagenumbering{arabic}
\setcounter{page}{11}


\chapter*{Introduction}
\markboth{}{Introduction}
\addcontentsline{toc}{chapter}{Introduction}

TODO 
\chapter{Game Introduction}
KOF '97 is a fighting game produced by SNK for arcade in 1997. Comparing with its pioneer Street Fighter 2, KOF '97  
\section{Command System}
\section{Combo System}
\chapter{Environment Setup for KOF97}

\section{Interact With Arched Emulator}


\section{Action Space}
\subsection{•}
\section{Observation Space}
\section{Reward System}
\chapter{Proposed network structure}


\chapter{Experiment}
\section{LSTM Model}

\section{Stacked features + CNN Model}
\chapter{Conclusion}
% ------------------------------- BIBLIOGRAPHY ---------------------------
% LEXICOGRAPHICAL ORDER BY AUTHORS' LAST NAMES
% FOR AMBITIOUS ONES - USE BIBTEX


\begin{thebibliography}{20} % IF YOU HAVE MORE REFERENCES, WRITE THE BIGGER NUMBER

\bibitem[1]{Ktos} A. Author, \emph{Title of a book}, Publisher, year, page--page.
\bibitem[2]{Innyktos} J. Bobkowski, S. Dobkowski, Title of an article, \emph{Magazine X, No. 7}, year, PAGE--PAGE.
\bibitem[3]{B} C. Brink, Power structures, \emph{Algebra Universalis 30(2)}, 1993, 177--216.
\bibitem[4]{H} F. Burris, H. P. Sankappanavar, \emph{A Course of Universal Algebra}, Springer-Verlag, New York, 1981.
\bibitem{8451491}Li, Y., Chang, H., Lin, Y., Wu, P. \& Wang, Y. Deep Reinforcement Learning for Playing 2.5D Fighting Games. {\em 2018 25th IEEE International Conference On Image Processing (ICIP)}. pp. 3778-3782 (2018)
\end{thebibliography}
\pagenumbering{gobble}
\thispagestyle{empty}



% ----------------------- LIST OF SYMBOLS AND ABBREVIATIONS ------------------
\chapter*{List of symbols and abbreviations}

\begin{tabular}{cl}
nzw. & nadzwyczajny \\
* & star operator \\
$\widetilde{}$ & tilde 
\end{tabular}
\\
If you don't need it, delete it.
\thispagestyle{empty}


% ----------------------------  LIST OF FIGURES --------------------------------
\listoffigures
\thispagestyle{empty}
If you don't need it, delete it.


% -----------------------------  LIST OF TABLES --------------------------------
\renewcommand{\listtablename}{Spis tabel}
\listoftables
\thispagestyle{empty}
If you don't need it, delete it.

% -----------------------------  LIST OF APPENDICES ---------------------------
\chapter*{List of appendices}
\begin{enumerate}
\item Appendix 1
\item Appendix 2
\item In case of no appendices, delete this part.
\end{enumerate}
\thispagestyle{empty}


\end{document}
